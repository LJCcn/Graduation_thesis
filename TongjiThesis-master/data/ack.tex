在此论文成稿之际,谨向我尊敬的导师苏永清副教授致以最诚挚的感谢!本文的选题内容、研究思路与分析过程均在苏老师的悉心指导下完成。
在两年的课题研究过程中,始终得到苏老师科研上的帮助与指导。苏老师渊博的学识、严谨的治学理念以及以身作则的工作态度,深深地感染
了我,在此我谨向您致以深切的感谢与诚挚的敬意。感谢岳继光教授与董延超副教授在工程项目及日常生活中给予我的帮助。你们严谨的学术
态度与踏实的工作作风为实验室建立了严谨求实的研究氛围,感谢你们为实验室的辛勤付出。

感谢同门师兄张鲲鹏与陈峰在研究课题中的指导,感谢刘志刚与徐晨剑师兄在工程项目中对我的帮助,徐晨剑师兄的论文~\LaTeX~模板使我受
益匪浅。感谢侯培鑫博士在学术方法上对我的提点以及多次赠书之情,感谢王森博博士与王栗博士在项目中的指导与帮助,祝你们顺利毕业。
感谢同届研究生施梁、汪嬴、唐丹旭、吴琛浩,同窗之情,友谊长存,祝你们今后工作顺利。感谢刘雪娇与穆慧华两位师妹在~502~研究所与
~811~研究所工程项目中的协助。感谢同济大学先进测控技术课题组徐刚、陈策、乔琪、张爽、武新然、李炅聪等全体成员,伴我度过了难忘的
研究生时光,在此祝大家前程似锦,幸福快乐。

同时感谢张志明老师对我在虚拟仪器俱乐部工作的支持与帮助,感谢黄蓉荣师姐、吉方成师兄与陈阳师兄在虚拟仪器俱乐部对我的指导与帮助,
祝同济大学虚拟仪器俱乐部越办越好。感谢贾青老师与同济大学~PACE~中心对我工作上的支持,感谢张世博师兄在~Urban Flexible Vehicle~项目
中的帮助,那些熬夜调车的夜晚至今历历在目。

此外,感谢好友狄宗林与邵瑶夏,与你们相识让我看到青年才俊们的努力与拼搏,祝你们最终实现梦想。感谢好友~Paolo Curti~在意大利的
热情招待,Omegna~湖畔的烟花绚烂依旧。


最后,要感谢我的父亲母亲,是你们的支持才使我能走到今天。养育之恩,无以为报,你们健康快乐是我最大的心愿。在未来的学习、
工作与生活中,我会继续锐意进取,成于精勤、止于至善。


~~\\


\rightline{赵闻达\qquad \quad}

\rightline{2018年3月于上海嘉定}