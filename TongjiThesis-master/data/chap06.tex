\chapter{总结与展望}
\label{cha:Sum}
\section{全文总结}
\label{sec:chap6:Sum}

为保证航空航天空间电源系统供电的稳定性,避免太空中复杂的电磁环境与温度脉冲对空间电源系统造成影响,导致电源系统性能变差,本文针对控制系统中脆弱性现象进行了清晰的定义,并采用数学方法建立了脆弱性量化评估体系,对空间电源系统进行脆弱性分析与量化评估,识别空间电源系统的薄弱环节。全文的主要工作总结如下:

(1) 查阅相关文献资料,对脆弱性概念的起源及发展进行了综述,阐述了脆弱性概念在控制领域的研究现状。通过分析控制系统稳定性、可靠性、鲁棒性以及脆弱性之间的区别与联系,结合控制系统中脆弱性问题的本质特征,得到本文中清晰的脆弱性概念。

(2) 根据本文中对控制系统脆弱性的定义,结合控制系统的不确定性、灵敏度分析以及鲁棒性理论,将控制系统脆弱性问题用数学方式描述。采用多指标综合评价相关理论,建立了一套针对控制系统不确定性的脆弱性分析与量化评估体系。

(3)针对本文的研究对象空间电源控制系统(Power Conditioning Unit, PCU),通过查阅航空航天领域文献资料,从极端温度环境与极端辐射环境两个方面研究,建立了航空航天领域中电阻、电容以及电感元件的数学模型,模拟空间电源系统在工作环境发生大幅变化时的参数变化。

(4) 依据本文所建立的控制系统脆弱性分析方法,以基于~Buck~变换器的空间电源系统为例,采用~Saber~与~MATLAB~软件进行模型仿真,分别对主拓扑电路中电容元件、电感元件以及电阻元件的不确定性进行脆弱性量化评估,明确系统脆弱性量化评估的方法与具体步骤,得到基于~Buck~变换器的空间电源系统脆弱性分析结果与系统薄弱环节。




%(3) 讨论了自抗扰控制技术在~PMSM~伺服系统转速环中和位置的应用。设计了转速环二阶自抗扰
%控制器,并对其性能进行了理论分析。通过仿真和实验,对比分析了转速环采用传统~PI~ 控制方式下系统的跟踪性能和抗扰性能。与~PI~相比,自抗扰控制器在响应速度和扰动抑制方面都具有更好的效果。
%
%(4)  使用~Popov~曲线法分析使用了~ADRC~ 控制器的~PMSM~矢量控制系统的稳定性,并求解稳定边界,为参数选择提供依据。



\section{未来工作展望}
\label{sec:chap6:Future}

由于本人水平与时间所限,对于航空航天空间电源系统的脆弱性分析与研究还存在需要完善及深入探讨之处,后续的工作可以从以下几个方面展开:

(1)  本文中脆弱性的本质体现在工作环境发生大幅变化时系统抵御不确定性的能力快速变差的特性。对于不同控制系统,抵御不确定性能力的评价指标不同。本文对~SISO~系统采用鲁棒稳定裕度的变化程度来评判系统的脆弱性,对于更加复杂的~MIMO~控制系统的脆弱性有待进一步研究。

(2) 本文分析与研究了控制系统中脆弱性概念所描述的实际问题,基于控制系统的不确定性与鲁棒性理论相关知识建立了脆弱性的数学描述方式,因此本文所定义的脆弱性需要建立在系统的不确定性基础上。而实际工程问题中存在一些由于系统结构所导致的对扰动敏感现象,本质上也属于脆弱性范畴。从系统零极点的角度定义与分析脆弱性同样值得深入研究。

(3)  由于实验条件有限,本文中通过查阅文献的方式建立了电子元件在极端辐射与极端温度环境下的数学模型,并根据该数学模型模拟电源系统在宇宙空间中的工作状态。若实验条件许可,可以对实际分析的电源电路进行高低温及辐射实验,通过实验得到的数据进行脆弱性分析,提高结果的可信性。

(4) 航空航天系统中设备繁多,工作环境复杂,电源系统可能出现多种不确定性同时产生的情况。本文分别对主拓扑电路中电容元件、电感元件以及电阻元件单独存在不确定性的情况进行脆弱性分析,并没有考虑三种不确定同时产生的情况。同时,对于实际空间电源系统其他环节存在的不确定性也可以采用本文建立的脆弱性量化分析方法进行分析与研究。