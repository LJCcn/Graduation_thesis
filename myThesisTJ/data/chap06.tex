\chapter{总结与展望}
\label{cha:summery}

\section{全文总结}
\label{sec:sum}
为保证风力发电系统产生的电能质量平稳满足用户的要求,避免地理气候等环境因素或用户负荷需求的变化对含风电电力系统造成影响,导致电能质量变差,本文针对含风电电力系统中脆弱性现象给出了科学合理的描述,并采用数学方法建立了脆弱性量化评估体系,对含风电电力系统进行脆弱性分析与量化评估,识别电力系统系统的薄弱环节。全文的主要工作总结如下:

(1)本文通过查阅相关的文献资料,对脆弱性概念的起源及发展进行了综述,并阐述了脆弱性概念在电力系统的研究现状。通过分析系统脆弱性存在的原因得出系统的脆弱性本质以及其数学描述,进而得出一个较为清晰的系统脆弱性概念。

(2)根据本文中对含风电电力系统脆弱性的定义,分别从系统的结构和系统的状态两个角度对脆弱性进行分析研究,得到系统的脆弱性理论。结构脆弱性方面,分别基于复杂网络和$PageRank$对系统拓扑进行分析。状态脆弱性方面,基于蒙特卡洛大量实验的方法将外界环境及用户端的随机性变化用实验模拟的方式进行分析。

(3)针对本文提出的系统脆弱性综合评估指标集,采用多指标综合评价相关理论,建立了一套关于含风电电力系统的脆弱性分析与量化评估体系。对不同的二级指标选择合适方法进行归一化后,分别采用层次分析法进行权重分配和指标融合。再使用$D-S$证据理论对一级指标进行融合,得到系统的综合脆弱性指标。

(4)依据本文所建立的含风电电力系统脆弱性分析方法,以$IEEE39$和$IEEE118$系统数据为例,采用~MATLAB~软件进行模型仿真,分别对单风机接入和多风机接入的情况下系统的结构脆弱性与状态脆弱性进行量化评估,明确系统脆弱性量化评估的方法与具体步骤,得到系统的脆弱性分析结果,识别出系统的薄弱环节。

\section{未来工作展望}
\label{sec:feature}
由于本人水平与时间所限,对于含风电电力系统的脆弱性分析与研究还存在需要完善及深入探讨之处,后续的工作可以从以下几个方面展开:

(1)本文定义的状态脆弱性只针对负荷节点,因为状态脆弱性主要分析的是潮流中节点电压的变化。而对于发电节点由于在潮流变化中本质上是$PV$节点,即发电节点的电压是不变的,所以无法用本文定义的状态脆弱性来衡量,后续可以从发电节点的无功功率的变化来分析其状态脆弱性。

(2)本文认为风力发电对电能质量的影响主要为支路潮流和节点电压的变化,实际上其影响还有很多,比如风力发电影响电能的频率。本文没有考虑含风电的系统的频率的变化,后续从频率的角度分析脆弱性同样值得深入研究。

(3)由于实验条件和时间的有限,本文中通过查阅文献的方式最终用概率统计的方式建立了随机性风电功率模型与负荷概率模型。若实验条件许可,可以对实际的电力系统所在地的风速、负荷需求进行采样统计,得到日变化或年变化的真实数据再用本文的方法进行系统的脆弱性量化分析,这样得到的结果更具现实意义。
