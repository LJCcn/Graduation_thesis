
在此论文成稿之际,谨向我尊敬的导师苏永清副教授致以最诚挚的感谢!本文的选题内容、研究思路与分析过程均在苏老师的悉心指导下完成。在两年的课题研究过程中,始终得到苏老师科研上的帮助与指导。苏老师渊博的学识、严谨的治学理念以及以身作则的工作态度,深深地感染了我,在此我谨向您致以深切的感谢与诚挚的敬意。感谢岳继光教授与董延超副教授在工程项目及日常生活中给予我的帮助。你们严谨的学术态度与踏实的工作作风为实验室建立了严谨求实的研究氛围,感谢你们为实验室的辛勤付出。

感谢同门师兄韩泽文、张鲲鹏、陈峰、赵闻达在研究课题中的指导,尤其是闻达师兄多次在项目上给予中肯的帮助与建议,带我走进$LabView$俱乐部,领略到了更多图形化编程语言的魅力。感谢侯培鑫博士在学术方法上对我的提点以及提供的论文~\LaTeX~模板使我受益匪浅。感谢王森博博士、吴琛浩博士与王栗博士在项目中的指导与帮助,祝你们顺利毕业。感谢同届研究生穆慧华、陈策、徐刚、张爽、乔琪,同窗之情,友谊长存,祝你们今后工作顺利。感谢李炅聪与武新然两位师弟在~502~研究所与~811~研究所工程项目中的协助。感谢同济大学先进测控技术课题组孙佳妮、林敏静、何士波、冀玲玲、何洪志、寿佳鑫、王浩天、宁少淳等全体成员,伴我度过了难忘的研究生时光,在此祝大家前程似锦,幸福快乐。

同时感谢张志明老师对我在虚拟仪器俱乐部工作的支持与帮助,感谢吉方成师兄、陈阳师兄、狄宗林部长、颜超进部长在虚拟仪器俱乐部对我的指导与帮助,祝同济大学虚拟仪器俱乐部越办越好。同时忠心感谢培育我的电子信息与工程学院,感谢每一位辛勤教学的老师。祝福你们一切顺利,也祝愿我的母校越来越好。

最后,要感谢我的家人,是你们的支持才使我能走到今天。养育之恩,无以为报,你们健康快乐是我最大的心愿。在未来的学习、工作与生活中,我会继续锐意进取,成于精勤、止于至善。


~~\\


%\rightline{刘雪娇~~~~~~~~~~~~~~}

\rightline{2018年12月}
