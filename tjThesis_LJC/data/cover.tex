\tongjisetup{
  %******************************
  % 注意:
  %   1. 配置里面不要出现空行
  %   2. 不需要的配置信息可以删除
  %******************************
  %
  %=====
  % 密级
  %=====
  secretlevel={保密},
  secretyear={2},
  %
  %=========
  % 中文信息
  %=========
  % 题目过长可以换行(推荐手动加入换行符,这样就可以控制换行的地方啦)。
  ctitle={基于结构与状态的电网脆弱性\\综合评估模型研究},
  cheadingtitle={基于结构与状态的电网脆弱性综合评估模型研究},    %用于页眉的标题,不要换行
  cauthor={李炅聪},
  studentnumber={1732929},
  cmajorfirst={工程},
  cmajorsecond={控制工程},
  cdepartment={电子与信息工程学院},
  csupervisor={苏永清~~~副教授},
  % 如果没有副指导老师或者校外指导老师,把{}中内容留空即可,或者直接注释掉。
  %cassosupervisor={裴刚 教授~(校外)}, % 副指导老师
  % 日期自动使用当前时间,若需手动指定,按如下方式修改:
  % cdate={\zhdigits{2018}年\zhnumber{11}月},
  % 没有基金的话就注释掉吧。
  %cfunds={(本论文由我要努力想办法撑到两行的著名国家杰出青年基金 (No.123456789) 支持)},
  %
  %=========
  % 英文信息
  %=========
  %etitle={Research on Vulnerability Analysis and \\Quantitative Evaluation of Power System \\Based on Complex Network},
  etitle={Research on Comprehensive Assessment Model \\of Power Grid Vulnerability \\Based on Structure and State},
  eauthor={Li Jiongcong},
  emajorfirst={Engineering},
  emajorsecond={Control Engineering},
  edepartment={College of Electronics and ~~~~~~~~~~~~~~~~Information Engineering},
%emajorfirst{Control Science and Engineering}
%emajorsecond{Control Theory and Control Engineering}
  % 日期自动使用当前时间,若需手动指定,按如下方式修改:
  % edate={November,\ 2018},
  %efunds={(Supported by the Natural Science Foundation of China for\\ Distinguished Young Scholars, Grant No.123456789)},
  esupervisor={A. Prof.  Su Yongqing},
  %eassosupervisor={Prof. Gang Pei (XiaoWai)}
  }

% 定义中英文摘要和关键字
\begin{cabstract}
电力系统作为维持国计民生的重要组成部分,其稳定性、可靠性及安全性至关重要。近年来,世界各地发生的多起大停电事故也引起越来越多专家学者的关注。研究表明,大多数大停电事故的发生原因是局部故障。
% 结构和状态是电网脆弱性研究的两个重要方面,电网结构的完整性是维持电网稳定运行的基础,电网状态的不稳定会导致局部故障的发生。
因此,综合评估和识别电网的脆弱环节并采取有效控制措施是避免大停电事故发生的关键。在此背景下,本文针对基于结构与状态的电网脆弱性综合评估模型展开研究,主要工作归纳如下:

% 在电力系统运行过程中,脆弱性是其固有属性,在扰动因素作用下,电网自身固有的脆弱性显现的概率大增,将对其稳定运行
% 造成威胁,因此有必要分析电力系统的脆弱性,识别电力系统的脆弱环节,科学合理地评估电网脆弱环节的脆弱性程度,对其进行结构优化和重点防护,保证电网稳定可靠运行。

本文研究了复杂网络特征参数和网络模型,验证分析电力系统的小世界性和无标度性。通过阐述复杂网络脆弱性概念得出,复杂网络的脆弱性在于其子系统的存在性,明确了本文研究的关键问题。

通过查阅各领域脆弱性相关文献,结合复杂网络脆弱性概念和电力系统脆弱性特征,得出较为清晰的电力系统脆弱性定义,并进行脆弱过程分析及数学描述。从结构和状态两个方面对系统脆弱性进行研究,
在结构方面,基于复杂网络理论建立电力系统拓扑模型,并提出结构脆弱性指标,建立了电网结构脆弱性模型;在状态方面,从节点电压稳定性、过负荷能力和电网损耗方面提出状态脆弱性指标,并建立电力
系统负荷模型,通过蒙特卡洛方法对状态脆弱性指标进行计算分析,建立了电网状态脆弱性模型。

建立电力系统脆弱性评估指标体系,分别选取反映系统脆弱性的结构和状态脆弱性指标进行分析,并对其归一化处理,采用改进熵权法和离差最大化法分别对结构脆弱性指标集和状态脆弱性指标集
进行权重分配和指标融合得到了脆弱性一级指标,进一步利用D-S证据理论对一级指标进行融合,建立了基于结构与状态的电网脆弱性综合评估模型。

最后,以$IEEE39$系统为研究对象,依据本文建立的电网脆弱性综合评估模型,
结合算例系统结构和参数分别对电力系统的结构和状态脆弱性进行分析,得到了结构脆弱性和状态脆弱性分析结果。最后,基于聚类算法和系统脆弱性综合评估结果进行脆弱节点等级评估,
根据得到的系统综合脆弱性等级评估结果,识别系统的脆弱环节。
% 针对电网结构脆弱性,本文制定了电网蓄意攻击策略,以$IEEE118$系统为研究对象,通过 MATLAB 算例仿真实验得出最优蓄意攻击策略,验证了结构脆弱性指标的合理性及重要性。
\end{cabstract}

\ckeywords{复杂网络,结构脆弱性,状态脆弱性,综合评估模型,脆弱环节识别}

\begin{eabstract}

\end{eabstract}

\ekeywords{} 