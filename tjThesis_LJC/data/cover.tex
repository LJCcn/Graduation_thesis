\tongjisetup{
  %******************************
  % 注意:
  %   1. 配置里面不要出现空行
  %   2. 不需要的配置信息可以删除
  %******************************
  %
  %=====
  % 密级
  %=====
  secretlevel={保密},
  secretyear={2},
  %
  %=========
  % 中文信息
  %=========
  % 题目过长可以换行(推荐手动加入换行符,这样就可以控制换行的地方啦)。
  ctitle={基于复杂网络的电网脆弱性\\分析与量化评估研究},
  cheadingtitle={基于复杂网络的电网脆弱性分析与量化评估研究},    %用于页眉的标题,不要换行
  cauthor={李炅聪},
  studentnumber={1732929},
  cmajorfirst={工程},
  cmajorsecond={控制工程},
  cdepartment={电子与信息工程学院},
  csupervisor={苏永清~~~副教授},
  % 如果没有副指导老师或者校外指导老师,把{}中内容留空即可,或者直接注释掉。
  %cassosupervisor={裴刚 教授~(校外)}, % 副指导老师
  % 日期自动使用当前时间,若需手动指定,按如下方式修改:
  % cdate={\zhdigits{2018}年\zhnumber{11}月},
  % 没有基金的话就注释掉吧。
  %cfunds={(本论文由我要努力想办法撑到两行的著名国家杰出青年基金 (No.123456789) 支持)},
  %
  %=========
  % 英文信息
  %=========
  etitle={Research on Vulnerability Analysis and \\Quantitative Evaluation of Power System \\Based on Complex Network},
  eauthor={Li Jiongcong},
  emajorfirst={Engineering},
  emajorsecond={Control Engineering},
  edepartment={College of Electronics and ~~~~~~~~~~~~~~~~Information Engineering},
%emajorfirst{Control Science and Engineering}
%emajorsecond{Control Theory and Control Engineering}
  % 日期自动使用当前时间,若需手动指定,按如下方式修改:
  % edate={November,\ 2018},
  %efunds={(Supported by the Natural Science Foundation of China for\\ Distinguished Young Scholars, Grant No.123456789)},
  esupervisor={A. Prof.  Su Yongqing},
  %eassosupervisor={Prof. Gang Pei (XiaoWai)}
  }

% 定义中英文摘要和关键字
\begin{cabstract}
电力系统作为维持国计民生的重要组成部分,其稳定性、可靠性及安全性至关重要。近年来,世界各地发生的多起大停电事故也引起越来越多专家学者的关注。从研究表明电网大停电事故的发生原因往往是局部故障
,由于电网的级联特性,最终导致大面积停电事故的发生。从电网大停电事故可以看出,在电力系统运行过程中,脆弱性是其固有属性,在扰动因素作用下,电网自身固有的脆弱性显现的概率大增,将对其稳定运行
造成威胁,因此有必要分析电力系统的脆弱性,识别系统脆弱环节,科学评估脆弱环节的脆弱性程度,对其进行结构优化和重点防护,保证电网稳定可靠运行。

$(1)$本文首先概述复杂网络特征参数和网络模型,验证和分析了电力系统的小世界性和无标度性。通过对复杂网络脆弱性概念的描述得知,复杂网络的脆弱性在于其子系统的存在性,进一步说明电力系统存在脆弱环节,
明确了本文电网脆弱性研究的关键问题。

$(2)$通过分析电力系统的稳定性、可靠性和鲁棒性的区别和联系,结合复杂网络脆弱性概念得出较为清晰的电力系统脆弱性定义,并进行脆弱过程分析及数学描述。分别从结构和状态两个方面对系统脆弱性进行分析,
在结构方面,基于复杂网络理论提出结构脆弱性指标,对电力系统拓扑建模;在状态方面,从电压稳定性、节点过负荷能力和电网损耗方面提出状态脆弱性指标,并建立电力系统负荷模型,通过蒙特卡洛方法进行状态
脆弱性分析。

$(3)$针对所提出的结构脆弱性指标和状态脆弱性指标,选取反映系统脆弱性的指标并进行归一化处理,采用改进熵权法和离差最大化法分别对结构脆弱性指标集和状态脆弱性指标集进行权重分配和数据融合得到脆弱性
一级指标,进一步利用D-S证据理论对一级指标进行融合得到系统综合评价结果。

$(4)$针对电网结构脆弱性,制定电网蓄意攻击策略,以$IEEE118$系统为研究对象,通过 MATLAB 算例仿真实验得出最优蓄意攻击策略,验证了结构脆弱性指标的合理性。以$IEEE39$系统为研究对象,结合系统结构
和系统参数分别得到了电力系统的结构脆弱性和状态脆弱性分析结果,最后,基于聚类算法和系统脆弱性综合评估结果进行脆弱节点等级评估,根据得到的脆弱性评估结果,识别系统的脆弱环节。



\end{cabstract}

\ckeywords{电网脆弱性,复杂网络理论,综合评价模型,脆弱环节识别}

\begin{eabstract}

\end{eabstract}

\ekeywords{} 