\tongjisetup{
  %******************************
  % 注意:
  %   1. 配置里面不要出现空行
  %   2. 不需要的配置信息可以删除
  %******************************
  %
  %=====
  % 密级
  %=====
  secretlevel={保密},
  secretyear={2},
  %
  %=========
  % 中文信息
  %=========
  % 题目过长可以换行(推荐手动加入换行符,这样就可以控制换行的地方啦)。
  ctitle={基于结构与状态的电网脆弱性\\综合评估模型研究},
  cheadingtitle={基于结构与状态的电网脆弱性综合评估模型研究},    %用于页眉的标题,不要换行
  cauthor={李炅聪},
  studentnumber={1732929},
  cmajorfirst={工程},
  cmajorsecond={控制工程},
  cdepartment={电子与信息工程学院},
  csupervisor={苏永清~~~副教授},
  % 如果没有副指导老师或者校外指导老师,把{}中内容留空即可,或者直接注释掉。
  %cassosupervisor={裴刚 教授~(校外)}, % 副指导老师
  % 日期自动使用当前时间,若需手动指定,按如下方式修改:
  % cdate={\zhdigits{2018}年\zhnumber{11}月},
  % 没有基金的话就注释掉吧。
  %cfunds={(本论文由我要努力想办法撑到两行的著名国家杰出青年基金 (No.123456789) 支持)},
  %
  %=========
  % 英文信息
  %=========
  %etitle={Research on Vulnerability Analysis and \\Quantitative Evaluation of Power System \\Based on Complex Network},
  etitle={Research on Comprehensive Assessment Model \\of Power Grid Vulnerability \\Based on Structure and State},
  eauthor={Li Jiongcong},
  emajorfirst={Engineering},
  emajorsecond={Control Engineering},
  edepartment={College of Electronics and ~~~~~~~~~~~~~~~~Information Engineering},
%emajorfirst{Control Science and Engineering}
%emajorsecond{Control Theory and Control Engineering}
  % 日期自动使用当前时间,若需手动指定,按如下方式修改:
  % edate={November,\ 2018},
  %efunds={(Supported by the Natural Science Foundation of China for\\ Distinguished Young Scholars, Grant No.123456789)},
  esupervisor={A. Prof.  Su Yongqing},
  %eassosupervisor={Prof. Gang Pei (XiaoWai)}
  }

% 定义中英文摘要和关键字
\begin{cabstract}
  电力系统作为维持国计民生的重要组成部分,其稳定性、可靠性及安全性至关重要,近年来,世界各地发生的多起大停电事故也引起越来越多专家学者的关注。研究表明大多数大停电事故的发生原因是局部故障,
  而综合评估和识别电网的脆弱环节并采取有效控制措施是避免大停电事故发生的关键,在此背景下,本文针对基于结构与状态的电网脆弱性综合评估模型展开研究。
  
  % 本文研究了复杂网络特征参数和网络模型,以$IEEE$30、39、57标准模型为例,验证分析了电力系统的小世界性和无标度性。通过研究复杂网络脆弱性得出电力系统存在脆弱环节,明确了本文研究的关键问题。
  
  本文以$IEEE$30、39、57标准模型为例,结合复杂网络脆弱性理论,分析了电力系统的小世界性和无标度性,明确了本文研究的关键问题,结合电力系统脆弱性特征,提出了电力系统脆弱性的定义,
  从结构和状态两个方面进行脆弱性模型研究和案例分析,在结构方面,基于复杂网络理论所建立的电网拓扑模型,提出结构脆弱性指标,建立了电网结构脆弱性模型;在状态方面,从节点电压稳定性、过负荷能力和
  电网损耗方面提出状态脆弱性指标,结合电力系统负荷概率模型,通过蒙特卡洛方法对状态脆弱性指标进行计算分析,建立了电网状态脆弱性模型。
  
  针对电网脆弱性综合量化评估问题,本文构建了电力系统脆弱性指标评估体系,选取并分析电网结构和状态脆弱性二级指标,对其进行归一化处理,在结构脆弱性方面,采用改进熵权法对结构脆弱性二级指标集
  进行权重分配得到结构脆弱性一级指标;在状态脆弱性方面,采用离差最大化法对状态脆弱性二级指标进行权重分配得到状态脆弱性一级指标,进而利用 D-S 证据理论对一级指标进行数据融合得到电网脆弱性综合
  评估指标,建立了基于结构与状态的电网脆弱性综合评估模型。
  
  最后,本文以 $IEEE39$ 标准算例模型为研究对象,依据电网脆弱性综合评估模型,分别对电力系统的结构和状态脆弱性进行分析,得到结构脆弱性和状态脆弱性评估结果,根据电网脆弱性综合指标,
  分析电网节点的综合脆弱性,验证了电网脆弱性综合评估模型的合理性。最后,基于聚类算法和电网脆弱性综合评估结果进行脆弱节点等级评估,根据得到的电网综合脆弱性等级评估结果,
  识别出电网的脆弱环节。

% 本文研究了复杂网络特征参数和网络模型,验证分析电力系统的小世界性和无标度性。通过阐述复杂网络脆弱性概念得出,复杂网络的脆弱性在于其子系统的存在性,明确了本文研究的关键问题。

% 通过查阅各领域脆弱性相关文献,结合复杂网络脆弱性概念和电力系统脆弱性特征,得出较为清晰的电力系统脆弱性定义,并进行脆弱过程分析及数学描述。从结构和状态两个方面对系统脆弱性进行研究,
% 在结构方面,基于复杂网络理论建立电力系统拓扑模型,并提出结构脆弱性指标,建立了电网结构脆弱性模型;在状态方面,从节点电压稳定性、过负荷能力和电网损耗方面提出状态脆弱性指标,并建立电力
% 系统负荷模型,通过蒙特卡洛方法对状态脆弱性指标进行计算分析,建立了电网状态脆弱性模型。

% 建立电力系统脆弱性评估指标体系,分别选取反映系统脆弱性的结构和状态脆弱性指标进行分析,并对其归一化处理,采用改进熵权法和离差最大化法分别对结构脆弱性指标集和状态脆弱性指标集
% 进行权重分配和指标融合得到了脆弱性一级指标,进一步利用D-S证据理论对一级指标进行融合,建立了基于结构与状态的电网脆弱性综合评估模型。

% 最后,以$IEEE39$系统为研究对象,依据本文建立的电网脆弱性综合评估模型,
% 结合算例系统结构和参数分别对电力系统的结构和状态脆弱性进行分析,得到了结构脆弱性和状态脆弱性分析结果。最后,基于聚类算法和系统脆弱性综合评估结果进行脆弱节点等级评估,
% 根据得到的系统综合脆弱性等级评估结果,识别系统的脆弱环节。
% % 针对电网结构脆弱性,本文制定了电网蓄意攻击策略,以$IEEE118$系统为研究对象,通过 MATLAB 算例仿真实验得出最优蓄意攻击策略,验证了结构脆弱性指标的合理性及重要性。
\end{cabstract}

\ckeywords{复杂网络,结构脆弱性,状态脆弱性,综合评估模型,脆弱环节识别}

\begin{eabstract}
  As an important part of maintaining national economy and people's livelihood, the stability, reliability and safety of electric power system are of vital importance.
  In recent years, more and more experts and scholars have paid more and more attention to the blackout around the world,the research shows that most of the causes of 
  blackouts are local faults, and the key to avoid blackouts is to comprehensively evaluate and identify the vulnerable links of the power grid and take effective control 
  measures. In this context, this thesis researchs the comprehensive evaluation model of power grid vulnerability based on structure and state.

  Taking $IEEE30$, 39, 57 standard power grid model as an example, combined with the theory of complex network vulnerability, this thesis analyzes the Small-World and Scale-Free 
  of power system, and clarifies the key issues of this thesis,combined with the characteristics of power system vulnerability, the definition of power system vulnerability is 
  proposed, and the vulnerability model research and case analysis are carried out from two aspects of structure and state.In the aspect of structure, based on the complex 
  network theory, the topological model of power grid is established, the structural vulnerability index is proposed, and the structural vulnerability model of power grid is 
  established, in the aspect of state, the state vulnerability index is proposed from the aspects of node voltage stability, overload capacity and power grid loss, combined 
  with the load probability model of power system, The state vulnerability index is calculated and analyzed by Monte Carlo method, and the state vulnerability model of power 
  grid is established.

  In view of the comprehensive quantitative assessment of power system vulnerability, this paper constructs a vulnerability index assessment system,select and analyze the 
  secondary indexes of power grid structure and state vulnerability, and normalize them,In the aspect of structural vulnerability, the improved entropy weight method is used 
  to allocate the weight of the secondary index set of structural vulnerability to obtain the primary index of structural vulnerability, in the aspect of state vulnerability, 
  the method of maximum deviation is used to allocate the weight of the secondary index of state vulnerability to obtain the primary index of state vulnerability, and then the 
  D-S evidence theory is used to fuse the primary index to obtain the comprehensive assessment index of power grid vulnerability model of power grid vulnerability based on 
  structure and state is established.

  Finally, this thesis takes the $IEEE39$ standard model as the research object, according to the comprehensive evaluation model of power grid vulnerability, 
  analyzes the structure and state vulnerability of power system respectively, obtains the evaluation results of structure vulnerability and state vulnerability, analyzes 
  the comprehensive vulnerability of power grid nodes according to the comprehensive index of power grid vulnerability, and verifies the rationality of the comprehensive 
  evaluation model of power grid vulnerability. Finally, based on the clustering algorithm and the comprehensive evaluation results of power grid vulnerability, the 
  vulnerable nodes are evaluated. According to the comprehensive evaluation results of power grid vulnerability, the vulnerable links of power grid are identified.

\end{eabstract}

\ekeywords{Complex network,Structural vulnerability,State vulnerability,Comprehensive assessment model,Identification of vulnerable links} 