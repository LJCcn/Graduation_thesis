\chapter{总结与展望}
\label{cha:summery}

\section{全文总结}
\label{sec:sum}
电力系统作为维持经济和社会发展的重要组成部分,其对于国家发展的重要性不言而喻。电网脆弱性已成为越来越多专家学者研究的热门领域,为此本文针对电力系统中脆弱性现象给出了科学合理的定义与数学描述,
从电网的结构和状态两个方面进行脆弱性模型研究,并采用并改进指标融合方法建立了基于结构与状态的脆弱性综合评估模型,对电力系统进行脆弱性分析与等级评估,识别了电力系统的脆弱环节,这对于设计
优化电网结构、降低电网大停电事故发生的概率等方面具有现实意义。全文的主要工作总结如下:

(1)通过查阅相关的文献资料,对电力系统脆弱性的发展及研究方法进行了综述。基于复杂网络理论验证分析了电力系统模型的小世界性和无标度性,通过研究复杂网络中局部与整体的关系,明确了本文
研究的关键问题。

(2)通过研究复杂系统脆弱性概念和电网脆弱性特征得出,电力系统的脆弱性本质在于其对内、外扰动的耐受程度,并从结构和状态两个方面研究电力系统脆弱性,进而得出较为清晰全面的系统脆弱性概念。

(3)根据对电力系统脆弱性的定义,进行结构和状态的脆弱性模型研究。在结构脆弱性方面,在扰动作用下,其侧重于电网结构保持完整性的能力和系统受影响的程度,基于复杂网络理论对电网系统进行建模,并提出结构脆弱性指标;
在状态脆弱性方面,本文在电压稳定性、承受负荷能力和电网损耗方面分别提出状态脆弱性指标,并建立随机负荷模型,结合潮流计算采用蒙特卡洛模拟实验方法计算分析电网的运行状态指标。

(4)针对提出的结构和状态方面的二级脆弱性指标,选择合适的数学方法对其进行无量纲归一化处理,然后采用改进熵权法和离差最大化法分别对结构和状态指标集进行权重分配,并得到结构和状态综合脆弱性指标,
最后采用D-S证据理论对结构和状态一级脆弱性指标进行融合得到系统综合脆弱性评价指标,建立了基于结构与状态的电网脆弱性综合评估模型。

(5)以$IEEE39$标准系统为例,依据本文所建立的电力系统脆弱性综合评估模型,使用MATLAB工具进行仿真实验,结合电网结构和系统状态参数,分别研究分析了系统结构
和状态脆弱性,根据D-S证据理论融合结构和状态脆弱性指标,进行电网综合脆弱性分析,最后,通过电网脆弱性综合评估结果对系统节点的脆弱程度进行了等级评估,识别出系统脆弱环节。

\section{未来工作展望}
\label{sec:feature}
由于本人水平与时间所限,对于电力系统脆弱性模型的研究还存在需要完善及深入探讨之处,后续的工作可以从以下几个方面展开:

(1)由于电网模型所限,本文在发电节点方面的脆弱性分析针对的只是结构方面,状态脆弱性指标只针对负荷节点,因为对于电网状态脆弱性是基于负荷变化进行研究的,而在潮流计算中发电节点为$PV$节点,
即发电节点的电压是不变的且有功功率为0,所以无法用本文定义的状态脆弱性指标来量化,后续可以采用其他电网模型通过暂态分析研究系统的状态脆弱性。

(2)本文在电网脆弱性研究方面,针对的只是系统节点,所以在识别系统脆弱环节方面,将节点的脆弱程度作为本文的研究重点,虽然电气介数指标考虑了支路对于节点的影响,但并未单独考虑系统支路对于
电力系统的影响,后续可以综合节点和支路两方面研究分析电网脆弱性。

(3)由于实验条件和时间的有限,本文中通过查阅文献的方式最终用概率统计的方式建立了随机性负荷概率模型来模拟实际负荷变化。若实验条件许可,可以对实际的电力系统所在地负荷需求进行采样统计,
得到日变化或年变化的真实数据再用本文所建立的模型进行系统的脆弱性量化分析,这样得到的评价结果更具现实意义。
