\chapter{电力系统脆弱性指标分析与量化评估方法}
\label{cha:quanti}

\section{引言}
\label{sec:index4}




\section{系统脆弱性量化评估指标}
\label{sec:describIndex}




\subsection{电力系统的脆弱性指标选取}
\label{sec:pickIndex}





\subsection{脆弱性量化评估指标的分析描述}
\label{sec:wordIndex}




\subsection{系统脆弱性综合评估指标集结构}
\label{sec:IndexSys}




\section{脆弱性量化评估二级指标融合}
\label{sec:processIndex}




\subsection{脆弱性量化评估二级指标归一化}
\label{sec:nomalzMethod}




\subsection{基于改进熵权法的权重分配}
\label{sec:nomalz}





\subsection{基于离差最大化法的权重分配}
\label{sec:nomalz}



\subsection{脆弱性量化评估二级指标融合}
\label{sec:2ndIndexMerge}





\section{脆弱性量化评估一级指标融合}




\subsection{$D-S$证据理论}
\label{sec:DStheory}





\subsection{脆弱性量化评估一级指标融合}
\label{sec:DSdistri}






\section{系统脆弱性量化评价模型描述}
\label{sec:systemQuan}





\section{本章小结}
\label{sec:sum4}
本章在系统脆弱性的研究基础上,根据脆弱性的定义与特征,针对结构脆弱性和状态脆弱性分别选取了3个能够反映脆弱性的指标,从而得到系统脆弱性综合评估指标集,其中对一级指标和二级指标进行了定义与区分。鉴于系统脆弱性问题的主观性,对系统脆弱性量化评估体系问题进行分析与研究。

针对脆弱性二级指标首先选择合适的方法进行归一化处理,然后采用改进熵权法和离差最大化进行权重分配,得到结构脆弱性和状态脆弱性的一级指标。最后,在$D-S$证据理论的研究基础上,将结构脆弱性指标与状态脆弱性指标进行融合,得到了系统综合脆弱性指标,为后文的脆弱性量化分析提供理论基础。



