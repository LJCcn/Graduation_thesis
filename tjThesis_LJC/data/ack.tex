两年多研究生的时光如白驹过隙,有些许不舍,也有许多感谢的话要说。

知遇之恩,铭刻在心。这里谨向我尊敬的导师苏永清副教授致以最诚挚的感谢!本文的选题方向、研究目标和研究思路均在苏老师的悉心指导下完成。在两年多的课题研究中,苏老师在科研和项目进展上对我帮助良多,
在研究生科研管理上,对我们多以鼓励和交流,给以宽松的科研环境,让我们减轻心理负担。另外,苏老师渊博的学识、严谨的治学理念以及以身作则的工作态度,深深地感染了我,在此再次向您致以深切的感谢与诚挚
的敬意。感谢岳继光教授与董延超副教授在工程项目及日常生活中给予我的帮助。你们严谨的学术态度与踏实的工作作风为实验室建立了严谨求实的研究氛围,感谢你们为实验室的辛勤付出。

良友在旁,见贤思齐。感谢同门师兄师姐赵闻达、刘雪娇、穆慧华在项目和课题研究中的指导,共同完成科研项目,带我走进 LabVIEW 俱乐部,领略到了更多图形化编程语言的魅力。感谢侯培鑫博士在学术方法
上对我的提点以及提供的论文 LATEX 模板使我受益匪浅,对实验室贡献良多。感谢王森博博士、吴琛浩博士、王栗博士在项目中的指导与帮助。感谢同届研究生武新然、刘金承、何士波、林敏静、孙佳妮,我们一起
共同进步,相互激励,同窗之情,友谊长存,祝你们今后工作顺利。
感谢同门寿佳鑫、黄靖斌师弟和吴宁馨师妹在项目和研究课题的帮助以及论文的校对工作,感谢同济大学先进测控技术课题组陈策、陈策、徐刚、张爽、乔琪、冀玲玲、王浩天、宁少淳、储承承、吴富潮等全体成员,
伴我度过了难忘的研究生时光,在此祝大家前程似锦,幸福快乐。感谢同届研究生董大亨、李武壮、涂康斌、张佳林,师弟郑宇、金易江、何洪志,时常一起打球、游泳,我们一直信奉强健的体魄才是学习工作的前提。
感谢同寝室的室友刘韶杰师弟,虽不在一学院,关系却十分融洽,祝你心想事成。
同时忠心感谢培育我的电子信息与工程学院,感谢每一位辛勤教学的老师。祝福你们一切顺利,也祝愿我的母校越来越好。

养育之恩。无以为报。感谢我的父母、家人,是你们的支持和鼓励才使我能走到今天,你们健康幸福是我最大的福报和心愿。在未来的学习、工作与生活中,我会继续坚持初心、砥砺前行。

~~\\

\rightline{李炅聪~~~~~~~~~~~}

\rightline{2019年12月}
