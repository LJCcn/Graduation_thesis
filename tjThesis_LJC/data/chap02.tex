\chapter{复杂网络相关概念}
\label{cha:model}

\section{引言}
\label{sec:index2}
经典的电力网络分析是基于基尔霍夫电流电压方程、固定的拓扑结构来进行分析的。其研究已形成了较为完善的体系\cite{refs60}。随着科学技术的不断发展,电力系统
变得越来越复杂、规模越来越大。在网络拓扑和分析计算上,经典的电力网络分析方法因节点和线路复杂度高的问题上已不再适用。而从复杂网络角度来看,电力系统可以
简化成一个由节点(母线)和边(支路)构成的网络。在复杂网络理论的基础上进行电网结构研究分析,已成为电网结构脆弱性分析的发展趋势。

电力系统作为典型的复杂网络,在研究电力系统脆弱性之初,有必要对复杂网络的相关概念进行阐述和分析。本章首先阐述复杂网络的基本概念,在此基础上分析其基本特征
参数。此外,基于复杂网络的建模方法,分析验证复杂网络的两类模型,小世界模型和无标度模型。最后,在基本特征参数的基础上,分析与描述复杂网络的脆弱性,并研究
其脆弱性常用指标,为后续章节的电力系统结构脆弱性分析实验打下基础。

\section{复杂网络基本概念}
\label{sec:powersys}
复杂网络的基本概念主要包括复杂网络的概念描述和统计描述,概念描述主要包括复杂网络语言描述和特点及特性描述。统计描述主要包括网络的基本特征参数和静
态特性,研究复杂网络的基本概念对分析后续电力系统脆弱性指标有着重要的意义。

\subsection{复杂网络概念描述}
\label{sec:composite}
复杂网络这一概念,人们试图严格去定义复杂系统,但直到现在还为充分认识和了解复杂网络,故难以给出其严格和实用的定义\cite{refs61}。目前普遍认同的是,钱学森
院士等人发起研究的系统科学研究领域--开放的复杂巨系统理论中对复杂网络的定义:具有自组织、自相似、吸引子、小世界、无标度中部分或全部性质的网络称为复杂网络
\footnote{摘自中文维基百科}。下面其性质进行具体阐述:

$(1)$自组织:对于这一概念,从不同学科角度来看会有不同的解读,我们从系统论的观点来看,“自组织”是指一个系统在内在机制的驱动下,自行从简单到复杂、从粗糙
向细致方向发展。不断提高自身的复杂度和精细度的过程。其与“他组织”的区别在于:“他组织”是靠外部指令而形成的组织,而“自组织”是系统按照内在机制自动形成的有序结构。
一个系统自组织属性愈强,其保持和产生新功能的能力也愈强。

$(2)$自相似:是指复杂网络的总体与部分、这一部分与另一部分之间的精细结构或性质所具有的相似性,或者可以这样理解:从整体中取出局部(局域)能够体现整体的基
本特征。

$(3)$吸引子:是微积分和系统科学论中的概念。一个系统有朝某个稳态发展的趋势,这个稳态叫做吸引子。吸引子是一个数学概念,描述系统运动的收敛类型,它存在于相平面。
简言之,吸引子是一个集合,当时间趋向于无穷大时,在任何一个有界集上出发的非定常流的所有轨迹都趋向于这个集合。

$(4)$小世界性:网络的小世界性是指对具有较短平均路径又具有较高聚类系数的网络性质的一种定义。平均路径长度和聚类系数会在下一小节展开描述。

$(5)$无标度性:网络的无标度性是指对网络的度分布符合幂律分布的复杂网络性质的一种定义。形象的来讲,复杂网络只有少数节点拥有大量的连接,而大部分节点却很少。




\subsection{复杂网络统计描述}
\label{sec:feature}



\section{复杂网络模型}
\label{sec:wind}


\subsection{小世界模型}
\label{sec:windEffects}



\subsection{无标度模型}
\label{sec:windModel}



\section{复杂网络的脆弱性描述}
\label{sec:load}



\subsection{脆弱性概念}
\label{sec:loadEffect}




\subsection{脆弱性常用指标}
\label{sec:loadModel}




\section{本章小结}
\label{sec:sum2}





